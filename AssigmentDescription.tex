\documentclass{scrreprt}
\usepackage[utf8]{inputenc}
\usepackage[hidelinks]{hyperref}
\usepackage{breakurl}
\usepackage{tabulary}
\usepackage{etoolbox}
\makeatletter
\patchcmd{\scr@startchapter}{\if@openright\cleardoublepage\else\clearpage\fi}{}{}{}
\makeatother

\author{Rick Rongen - 316789 - 2502968}
\title{DS-ML Mini-Project - Diamonds}
\date{\today}

\begin{document}
	\maketitle
	\chapter{Dataset Choses}
		The dataset I'm planning to use is called Diamonds and can be find on Kaggle using the following link: \url{https://www.kaggle.com/shivam2503/diamonds}\par
		
		This dataset describes all kinds of different diamonds with their quality (cut, color, carat and clarity) and size (x, y, z, dept and table). Also it describes the price of the diamond. I would like to use different machine learning techniques to calculate the price of a diamond given one or more of these measurements. Also I would like to be able to do the reverse and see what quality a diamond may be given it's size and price.
		
	\chapter{Example Data}
		In Pandas the cut, color and clarity column will be separated into separate columns or be represented as a number.\par
		Example data:
		\begin{table}[ht]
			\centering
			\caption{Data}
			\label{table-data}
			\begin{tabular}{|l|l|l|l|l|l|l|l|l|l|l|}
				\hline id & carat & cut       & color & clarity & depth & table & price & x    & y    & z    \\ \hline
				1  & 0.23  & Ideal     & E     & SI2     & 61.5  & 55    & 326   & 3.95 & 3.98 & 2.43 \\ \hline
				2  & 0.21  & Premium   & E     & SI1     & 59.8  & 61    & 326   & 3.89 & 3.84 & 2.31 \\ \hline
				3  & 0.23  & Good      & E     & VS1     & 56.9  & 65    & 327   & 4.05 & 4.07 & 2.31 \\ \hline
				4  & 0.29  & Premium   & I     & VS2     & 62.4  & 58    & 334   & 4.2  & 4.23 & 2.63 \\ \hline
				5  & 0.31  & Good      & J     & SI2     & 63.3  & 58    & 335   & 4.34 & 4.35 & 2.75 \\ \hline
				6  & 0.24  & Very Good & J     & VVS2    & 62.8  & 57    & 336   & 3.94 & 3.96 & 2.48 \\ \hline
				7  & 0.24  & Very Good & I     & VVS1    & 62.3  & 57    & 336   & 3.95 & 3.98 & 2.47 \\ \hline
				8  & 0.26  & Very Good & H     & SI1     & 61.9  & 55    & 337   & 4.07 & 4.11 & 2.53 \\ \hline
				9  & 0.22  & Fair      & E     & VS2     & 65.1  & 61    & 337   & 3.87 & 3.78 & 2.49 \\ \hline
				10 & 0.23  & Very Good & H     & VS1     & 59.4  & 61    & 338   & 4    & 4.05 & 2.39 \\ \hline
				11 & 0.3   & Good      & J     & SI1     & 64    & 55    & 339   & 4.25 & 4.28 & 2.73 \\ \hline
				12 & 0.23  & Ideal     & J     & VS1     & 62.8  & 56    & 340   & 3.93 & 3.9  & 2.46 \\ \hline
				13 & 0.22  & Premium   & F     & SI1     & 60.4  & 61    & 342   & 3.88 & 3.84 & 2.33 \\ \hline
				14 & 0.31  & Ideal     & J     & SI2     & 62.2  & 54    & 344   & 4.35 & 4.37 & 2.71 \\ \hline
			\end{tabular}			
		\end{table}
	
		\section{Table Description}
		\begin{tabulary}{\linewidth}{lL}
			carat & Carat weight of the diamond \\\hline
			cut & The cut quality of the diamond. Quality in increasing order Fair, Good, Very Good, Premium, Ideal \\\hline
			color & Color of the diamond, with D being the best and J the worst \\\hline
			clarity & How obvious inclusions are within the diamond:(in order from best to worst, FL = flawless, I3= level 3 inclusions) FL,IF, VVS1, VVS2, VS1, VS2, SI1, SI2, I1, I2, I3 \\\hline
			depth & Total depth percentage = z / mean(x, y) = 2 * z / (x + y) \\\hline
			table & Width of top of diamond relative to widest point  \\\hline
			price & The price of the diamond \\\hline
			x & Length mm \\\hline
			y & Width mm \\\hline
			z & Depth mm \\\hline
		\end{tabulary}
\end{document}